\section{Manual de Usuario}
\subsection{Introducción}
En esta sección se detallan los pasos a seguir para el uso adecuado del software Detector de Pivotes de Riego y Silo bolsas en Imágenes Satelitales para aplicaciones agrícolas.

\subsection{Pre-requisitos}
Contar con Docker instalado en la computadora.

En el caso de no optar por el uso de Docker, se puede correr localmente de contar con los siguientes requerimientos:
\begin{itemize}
    \item Distribución Linux: Debian o Ubuntu preferentemente (o sus derivados).
    \item Python 3.7 o superior.
    \item Virtualenv.
\end{itemize}

\subsection{Instalación}
\begin{enumerate}
    \item Descargar o clonar el repositorio desde Github: \url{https://github.com/gianfrancob/detector_pivotes_silobolsas}
    \item Inicializar el contenedor Docker corriendo: 
    \begin{itemize}
            \item \textit{docker build -t detector\_pivotes\_silobolsas }
            \item \textit{docker run -p 5000:5000 --name test -it detector\_pivotes\_silobolsas}
    \end{itemize}
    
    \item En el caso de no optar por el uso de Docker, se puede correr localmente de la siguiente manera:
    \begin{itemize}
        \item Crear entorno virtual usando virtualenv: \textit{virtualenv venv}
        \item Activar el entorno: \textit{source ./venv/bin/activate}
        \item Instalar las dependencias en el entorno: \textit{pip install -r ./requirements.txt}
    \end{itemize}
    \item Iniciar el servicio backend: \textit{python ./utils/flask\_rest\_api/restapi\_pivot\_silobolsa.py --port 5000}
    \item Ingresar a la página abriendo en el navegador el siguiente fichero HTML: \textit{./utils/boostrap\_frontend/index.html}
\end{enumerate}

xt

\subsection{Ingresar a la página}
Para ejecución local, la dirección URL donde se encuentra la aplicación es: 
\textit{DIRECTORIO RAIZ/utils/boostrap\_frontend/index.html}

Para ejecución web, dependerá particularmente del tipo de despliegue implementado.

\subsection{Cargar una imagen}

\begin{figure}[b!]
    \centering
    \includegraphics[width=0.8\textwidth]{img/FE - upload file.png}
    \caption{Subir una imagen para analizar}
    \label{fig:subir imagen}
\end{figure}

Una vez en la página principal, dirigirse al botón "Seleccionar archivo" que se encuentra en la mitad de la misma y hacer clic en él para seleccionar una imagen o desde la computadora del usuario, los formatos soportados son .jpg, .png .

\subsubsection{Error en el formato de la imagen}

Si el formato de la imagen cargada esta dañado o es invalido, se mostrara el siguiente mensaje, como se muestra en la figura \ref{fig:error}:

\begin{figure}[h!]
    \centering
    \includegraphics[width=0.6\textwidth]{img/FE - upload error.png} 
    \caption{Error al subir una archivo de formato invalido}
    \label{fig:error}
\end{figure}

\subsection{Cargar varias imágenes} 

Para la opción del análisis de un conjunto de imágenes, previamente se debe crear un archivo de compresión con todas las imágenes en alguno de los siguientes formatos: RPM (.rpm), RAR (.rar), RZIP (.rz), TAR (.tar), XZ (.xz), ZIP (.zip, .jar), GZIP (.gz), 7z (.7z).

Luego, hacer clic en el botón "Seleccionar archivo" para subir el archivo con todas las imágenes a ser analizadas.

\textbf{Nota}: No debe salir ningún mensaje de error para continuar con el análisis,  ver imagen \ref{fig:error}

\subsection{Inicio de la detección} 

Una vez cargada la imagen o las imágenes en formatos validos, hacer clic en el botón \textbf{"Submit"}, para dar inicio a la detección. 

\begin{figure}[h!]
    \centering
    \includegraphics[width=0.6\textwidth]{img/FE - detection loading.png}
    \caption{Inicio de la detección - Loading}
    \label{fig:loading}
\end{figure}

El botón cambiara de nombre y aparece como "Loading", indicando que la detección esta en proceso. Como se puede apreciar en la Figura \ref{fig:loading}.


\subsection{Descarga de imágenes}

Una vez realizada la detección de los objetos, se va a mostrar como resultado las imágenes analizadas indicando la presencia de los objetos Silo bolsas o Riegos por Pívot junto con una tabla donde se detalla la información de la detección.

Luego, se habilitara un botón \textbf{"Download"} que al hacer clic en él, se procede a la descarga de la/s imágenes, como se ve en la Figura \ref{fig:bulk detection} y se debe seleccionar el archivo del destino como se muestra en la Figura \ref{fig:download file}

\begin{figure}[h!]
    \centering
    \includegraphics[width=0.93\textwidth]{img/FE - bulk detection.png}
    \caption{Detección de un archivo con imágenes comprimidas}
    \label{fig:bulk detection}
\end{figure}

\begin{figure}[h!]
\centering
    \includegraphics[width=1\textwidth]{img/FE - download file.png}
    \caption{Descarga de archivo con imágenes detectadas - Resultados}
    \label{fig:download file}
\end{figure}

\pagebreak

% Los manuales se elaboran en base a las necesidades de los usuarios de la empresa a la que se le ha realizado el sistema. Algunos pueden contener los siguientes apartados:

% Portada
% Índice
% Introducción
% Instalación del sistema
% Diagrama general del sistema
% Diagrama particular detallado
% Explicación genérica de las fases del sistema
% Iniciación al uso del sistema
% Manual de Referencia


% Las secciones de un manual de usuario2​ a menudo incluyen:

% Una página de portada.
% Una página de título y una página de derechos de autor.
% Un prefacio, que contiene detalles de los documentos relacionados y la información sobre cómo navegar por el manual del usuario.
% Una sección de propósito, que es más una descripción general que un resumen del objetivo del documento.
% Una sección de audiencia que indique explícitamente quién es la audiencia prevista a leer el manual, incluyendo la audiencia opcional.
% Una sección de alcance es crucial ya que también sirve como un descargo de responsabilidad, indicando lo que está fuera del alcance y lo que sí está cubierto.
% Una guía sobre cómo utilizar al menos las principales funciones del sistema, es decir, sus funciones básicas.
% Una sección de solución de problemas que detalla los posibles errores o problemas que pueden surgir, junto con la forma de solucionarlos.
% Una sección de preguntas frecuentes, donde encontrar más ayuda y datos de contacto.
% Un glosario y, para documentos más grandes, un índice.
